\subsection*{Feuille n°1 — Sequences et intégrales}

\subsubsection*{Suites numériques}

\textbf{Exercice 1 (Moyenne de Césaro).}
Soit $(a_n)_{n\ge 1}$ une suite complexe convergente, de limite $\ell$.
\begin{enumerate}
	\item Montrer que la suite $(b_n)_{n\ge 1}$ définie par
	\[
	\forall n\ge 1,\qquad b_n=\frac{a_1+\cdots+a_n}{n}
	\]
	converge vers $\ell$.
	\item Montrer plus généralement que pour toute suite de réels positifs $(\varepsilon_n)$ telle que la série
	\[
	\sum_{n\ge 1}\varepsilon_n
	\]
	diverge, on a
	\[
	\lim_{n\to+\infty}\frac{a_1\varepsilon_1+\cdots+a_n\varepsilon_n}{\varepsilon_1+\cdots+\varepsilon_n}=\ell.
	\]
\end{enumerate}

\textbf{Exercice 2.}
Soit $(u_n)$ une suite définie par
\[
u_0>0,\quad u_1>0,\quad \lambda>0,\quad \forall n\in\mathbb{N},\qquad
u_{n+2}=\lambda\sqrt{u_{n+1}u_n}.
\]
\begin{enumerate}
	\item Montrer que la suite $(u_n)_{n\ge 1}$ est bien définie.
	\item Soit $(v_n)$ la suite définie par $v_n=\ln(u_n)$. Chercher une solution particulière $(w_n)$ de $(v_n)$ sous la forme $w_n=\alpha^n$.
	\item En déduire une récurrence linéaire à coefficients constants et expliciter le $n$-ième terme $u_n$ en fonction de $n$.
\end{enumerate}

\textbf{Exercice 3.}
Soit $\alpha>0$ un nombre irrationnel et $(r_n)$ une suite de nombres rationnels qui converge vers $\alpha$.
Pour tout $n$, on écrit
\[
r_n=\frac{p_n}{q_n},\qquad p_n\in\mathbb{Z},\quad q_n\in\mathbb{N}^*.
\]
Montrer que
\[
\lim_{n\to+\infty}p_n=+\infty
\qquad\text{et}\qquad
\lim_{n\to+\infty}q_n=+\infty.
\]

\textbf{Exercice 4.}
Soit $(u_n)$ une suite telle que
\[
u_0>0,\qquad \forall n\in\mathbb{N},\qquad u_{n+1}=\sqrt{\frac{1+u_n}{2}}.
\]
Prouver que la suite $(v_n)$ définie par
\[
\forall n\in\mathbb{N},\qquad v_n=\prod_{i=1}^{n}u_i
\]
est convergente et calculer sa limite.

\medskip
\textit{Rappel de formule trigonométrique :}
\[
\forall x\in\mathbb{R},\qquad \cos\!\left(\frac{x}{2}\right)=\sqrt{\frac{1+\cos(x)}{2}}.
\]

\subsubsection*{Intégrales}

\textbf{Exercice 1.}
Soient $[a,b]$ un segment de $\mathbb{R}$ non réduit à un singleton et $f:[a,b]\to\mathbb{C}$ une fonction de classe $C^1$ telle que $f(a)=0$.
\begin{enumerate}
	\item Montrer que
	\[
	\int_a^b |f(x)|^2\,dx\ \le\ \frac{(b-a)^2}{2}\int_a^b |f'(x)|^2\,dx.
	\]
	\item Montrer que
	\[
	\left|\int_a^b f'(x)f(x)\,dx\right|
	\ \le\ \frac{b-a}{2}\int_a^b |f'(x)|^2\,dx.
	\]
	\item Caractériser le cas d’égalité dans les deux inégalités précédentes.
\end{enumerate}

\textbf{Exercice 2.}
Soit $f:\mathbb{R}_+\to\mathbb{R}$ une fonction de classe $C^1$, intégrable sur $\mathbb{R}_+$, telle que $(f')^2$ soit intégrable sur $\mathbb{R}_+$. Montrer que $f$ est bornée sur $\mathbb{R}_+$.

\medskip
\textit{Indication :} utiliser le théorème suivant.

\medskip
\textbf{Théorème.}
Soient $a\in\mathbb{R}$ et $E$ un $\mathbb{K}$-espace vectoriel normé ($\mathbb{K}=\mathbb{R}$ ou $\mathbb{C}$).
Soit $f:[a,+\infty[\to E$ une fonction uniformément continue sur $[a,+\infty[$.
Si
\[
\int_a^{+\infty} f(t)\,dt
\]
converge, alors
\[
\lim_{t\to+\infty} f(t)=0.
\]

\medskip
\textit{Rappel :} une fonction $f:E\to E$ est dite uniformément continue si
\[
\forall \varepsilon>0,\ \exists \delta>0,\ \forall(x,y)\in E^2,\ d(x,y)\le\delta\ \Rightarrow\ d(f(x),f(y))\le\varepsilon.
\]


