\section*{Séquences}

\subsection*{Exercice 1}
1. 

D'abbord, comme la suite $(a_n)$ est bornée, on peut définir $D = \sup_n |a_n - l|$.

Ensuite, puisque $(a_n)$ converge vers $l$, pour tout $\epsilon > 0$, il existe $N_1 \in \mathbb{N}$ tel que pour tout $n \geq N$, on a $|a_n - l| < \frac{\epsilon}{2}$.

Donc,

$$
\left|b_n - l\right| = \left|\frac{1}{n}\sum_{k=1}^n (a_k - l)\right| \leq \frac{1}{n}\sum_{k=1}^n |a_k - l| = \frac{1}{n}\left(\sum_{k=1}^{N_1} |a_k - l| + \sum_{k=N_1+1}^n |a_k - l|\right) \leq \frac{1}{n}\left(N_1 D + (n-N_1)\frac{\epsilon}{2}\right)
$$
\\
Or, $\frac{N_1 D}{n} \to 0 \text{ quand } n \to \infty, \text{ donc  }$ $\exists N_2 \in \mathbb{N}$ tel que pour tout $n \geq N_2$, on a $\frac{N_1 D}{n} < \frac{\epsilon}{2}$.
\\
Ainsi, on prend $N = \max(N_1, N_2)$ et pour tout $n \geq N$, on a $|b_n - l| < \epsilon$. $\qed$
\\

2. 

Soit $(\varepsilon_n)$ une suite de réels positifs telle que la série $\sum_{n \geq 1} \varepsilon_n$ diverge. On note $S_n = \varepsilon_1 + \cdots + \varepsilon_n$.

On a que, 

$$
\lvert \frac{\sum_{k=1}^n a_k\varepsilon_k}{\sum_{k=1}^n \varepsilon_k} - l \rvert = |\frac{\sum_{k=1}^n (a_k - l)\varepsilon_k}{S_n}| \leq \frac{1}{S_n}\left(\sum_{k=1}^{N_1} |a_k - l|\varepsilon_k + \sum_{k=N_1+1}^n |a_k - l|\varepsilon_k\right)
$$

Or, puisque $(a_n)$ converge vers $l$, il existe $N_1 \in \mathbb{N}$ tel que pour tout $k \geq N_1 + 1$, on a $|a_k - l| < \epsilon$, pour tout $\epsilon > 0$.

Ainsi, on a que $\sum_{k=1}^{N1}\varepsilon_k |a_k - l| \leq M$ pour une certaine constante $M > 0$.

Alors, $$\lvert \frac{\sum_{k=1}^n a_k\varepsilon_k}{\sum_{k=1}^n \varepsilon_k} - l \rvert \leq \frac{1}{S_n}\left(M + \sum_{k=N_1+1}^n |a_k - l|\varepsilon_k\right) \leq \frac{M}{S_n} + \frac{1}{S_n}\sum_{k=N_1+1}^n |a_k - l|\varepsilon_k \leq \frac{M}{S_n} + \epsilon$$

Et puisque la série $\sum_{n \geq 1} \varepsilon_n$ diverge, on a que $S_n \to +\infty$ quand $n \to +\infty$. Alors, $\frac{M}{S_n} \to 0$ et donc,

$$
\lim_{n \to +\infty} \frac{\sum_{k=1}^n a_k\varepsilon_k}{\sum_{k=1}^n \varepsilon_k} = l. \qed
$$

\subsection*{Exercice 2}

1.

Puisque $u_0 > 0, u_1>0$ et $\lambda<1$, on a $u_2 = \lambda \sqrt{u_1 u_0} >0$. On suppose que $u_j > 0$ pour tout $j < n$. Alors, $u_n = \lambda \sqrt{u_{n-1} u_{n-2}} > 0$, donc par récurrence, $u_n > 0$ pour tout $n \in \mathbb{N}$. \\

2.

Puisque $u_n > 0$ pour tout $n$, on peut définir $v_n = \ln(u_n)$. Ainsi,

$$
v_n = \ln(u_n) = \ln(\lambda \sqrt{u_{n-1} u_{n-2}}) = \ln(\lambda) + \frac{1}{2}(v_{n-1} + v_{n-2})  (*)
$$


Soit $w_n$ une solution de (*) sous la forme $w_n = \alpha n$, on a
$$\alpha n = \ln(\lambda) + \frac{1}{2}(\alpha (n-1) + \alpha (n-2))$$
$$\Longleftrightarrow \alpha n = \ln(\lambda) + \alpha n - \frac{3}{2}\alpha$$
$$\Longleftrightarrow \alpha = -\frac{2 \ln(\lambda)}{3}$$ et donc,

$$w_n = -\frac{2 \ln(\lambda)}{3} n$$ \\

3. 

Comme $u_n>0$ pour tout $n$ (question 1), on peut prendre le logarithme dans la relation
\[
u_{n+2}=\lambda\sqrt{u_{n+1}u_n}.
\]
On obtient, pour tout $n\in\mathbb{N}$,
\[
v_{n+2}=\ln(u_{n+2})
=\ln(\lambda)+\ln\!\Big(\sqrt{u_{n+1}u_n}\Big)
=\ln(\lambda)+\frac12\ln(u_{n+1})+\frac12\ln(u_n).
\]
Donc
\[
\boxed{\; v_{n+2}-\frac12v_{n+1}-\frac12v_n=\ln(\lambda)\; } \qquad (\star)
\]
qui est une récurrence linéaire affine à coefficients constants.

L'équation homogène associée est
\[
v_{n+2}-\frac12v_{n+1}-\frac12v_n=0.
\]
On cherche une solution sous la forme $v_n=r^n$, d'où l'équation caractéristique
\[
r^2-\frac12 r-\frac12=0
\quad\Longleftrightarrow\quad
2r^2-r-1=0.
\]
Ses racines sont
\[
r_1=1,\qquad r_2=-\frac12.
\]
Ainsi, la solution générale de l'homogène est
\[
v_n^{(h)}=A+B\Big(-\frac12\Big)^n,
\]
où $A,B\in\mathbb{R}$.

Le second membre de $(\star)$ est constant, mais comme $r=1$ est racine de l'équation caractéristique,
une particulière constante ne convient pas ; on essaie donc une solution particulière affine $v_n^{(p)}=cn$.

En substituant dans $(\star)$ :
\[
c(n+2)-\frac12c(n+1)-\frac12cn
= c(n+2)-\frac{c}{2}(2n+1)
= c\Big(2-\frac12\Big)
= \frac32c.
\]
On veut $\frac32c=\ln(\lambda)$, donc
\[
c=\frac{2}{3}\ln(\lambda).
\]
Ainsi, une solution particulière est
\[
v_n^{(p)}=\frac{2}{3}\,n\ln(\lambda).
\]

En combinant homogène + particulière :
\[
\boxed{\; v_n = A + B\Big(-\frac12\Big)^n + \frac{2}{3}\,n\ln(\lambda)\; }.
\]
Comme $u_n=e^{v_n}$, on obtient
\[
u_n=\exp\!\Big(A + B\Big(-\frac12\Big)^n\Big)\cdot
\exp\!\Big(\frac{2}{3}\,n\ln(\lambda)\Big)
=\exp\!\Big(A + B\Big(-\frac12\Big)^n\Big)\,\lambda^{\frac{2n}{3}}.
\]
En posant $C=e^A>0$, cela s'écrit
\[
\boxed{\; u_n = C\,\lambda^{\frac{2n}{3}}\exp\!\Big(B\Big(-\frac12\Big)^n\Big)\; }.
\]

On a
\[
v_0=\ln(u_0)=A+B,
\qquad
v_1=\ln(u_1)=A-\frac{B}{2}+\frac{2}{3}\ln(\lambda).
\]
Donc
\[
\frac32B = \ln(u_0)-\ln(u_1)+\frac{2}{3}\ln(\lambda)
\quad\Longrightarrow\quad
\boxed{\; B=\frac{2}{3}\Big(\ln\!\frac{u_0}{u_1}+\frac{2}{3}\ln(\lambda)\Big)\; }.
\]
Puis
\[
\boxed{\; A=\ln(u_0)-B \;}
\quad\text{(et donc } C=e^A=u_0e^{-B}\text{)}.
\]

\subsection*{Exercice 3}

Montrons que $q_n\to +\infty$.
Supposons par l'absurde que $(q_n)$ ne tende pas vers $+\infty$.
Alors $(q_n)$ est bornée : il existe $M\ge 1$ tel que
\[
\forall n\in\mathbb{N},\qquad 1\le q_n\le M.
\]
Ainsi, $(q_n)$ ne prend ses valeurs que dans l'ensemble fini $\{1,2,\dots,M\}$.
Par conséquent, il existe $q\in\{1,\dots,M\}$ et une sous-suite $(q_{n_k})_{k\ge 1}$ telle que
\[
\forall k\ge 1,\qquad q_{n_k}=q.
\]
On considère la sous-suite correspondante :
\[
r_{n_k}=\frac{p_{n_k}}{q}.
\]
Comme $r_n\to\alpha$, on a aussi $r_{n_k}\to\alpha$. En multipliant par $q$, il vient
\[
p_{n_k}=q\,r_{n_k}\xrightarrow[k\to\infty]{} q\alpha.
\]
Or $q\in\mathbb{N}^*$ et $\alpha$ est irrationnel, donc $q\alpha$ est irrationnel.
Mais $(p_{n_k})$ est une suite d'entiers, ce qui est impossible puisqu'une suite d'entiers
convergente converge nécessairement vers un entier.
Contradiction. Donc
\[
q_n\xrightarrow[n\to\infty]{} +\infty.
\]

Montrons que $p_n\to +\infty$.
Puisque $r_n\to\alpha$ et $\alpha>0$, en prenant $\varepsilon=\alpha/2$, il existe $n_0\in\mathbb{N}$ tel que,
pour tout $n\ge n_0$,
\[
|r_n-\alpha|<\frac{\alpha}{2}
\quad\Longrightarrow\quad
\frac{\alpha}{2}<r_n<\frac{3\alpha}{2}.
\]
Comme $q_n>0$, on peut multiplier l'inégalité par $q_n$ :
\[
\forall n\ge n_0,\qquad
\frac{\alpha}{2}\,q_n<p_n<\frac{3\alpha}{2}\,q_n.
\]
En particulier,
\[
\forall n\ge n_0,\qquad p_n>\frac{\alpha}{2}\,q_n.
\]
Or $q_n\to+\infty$, donc $(\alpha/2)q_n\to+\infty$, et par comparaison on obtient
\[
p_n\xrightarrow[n\to\infty]{} +\infty.
\]

Ainsi,
\[
\lim_{n\to\infty} q_n=+\infty
\qquad\text{et}\qquad
\lim_{n\to\infty} p_n=+\infty.
\]

\subsection*{Exercice 4}

(i) Soint $u_0 = 0 \Rightarrow u_n = 1 \forall n \in \N$ par récurrence et donc $v_n \to 1$

(ii) Soit $u_0 < 1$, par recurrence $u_n < 1 \forall n \in \N$, puisque $u_n < 1 \Rightarrow \frac{1+u_n}{2} < 1 \Rightarrow u_{n+1} = \sqrt{\frac{1+u_n}{2}} < 1$

Alors, soit $\theta$ tel que $\cos\theta = u_0$. Ainsi, $u_n = \cos(\frac{\theta}{2^n})$ puisque $\forall x \in \R \cos(\frac{x}{2}) = \sqrt{\frac{1+\cos x}{2}}$

$$
v_n = \prod_{k=1}^n \cos\left(\frac{\theta}{2^k}\right) \Rightarrow 2^n \sin\left(\frac{\theta}{2^n}\right)v_n = \sin(\theta) \Rightarrow v_n = \frac{\sin(\theta)}{2^n \sin\left(\frac{\theta}{2^n}\right)} \to \frac{\sin(\theta)}{\theta} 
$$

Donc $v_n \to \frac{\sin(\theta)}{\theta}$ où $\theta = \arccos(u_0)$.

(iii) Soit $u_0 > 1$, par recurrence $u_n > 1 \forall n \in \N$, puisque $u_n > 1 \Rightarrow \frac{1+u_n}{2} > 1 \Rightarrow u_{n+1} = \sqrt{\frac{1+u_n}{2}} > 1$

Alors, soit $\theta$ tel que $\cosh\theta = u_0$. Ainsi, $u_n = \cosh(\frac{\theta}{2^n})$ puisque $\forall x \in \R \cosh(\frac{x}{2}) = \sqrt{\frac{1+\cosh x}{2}}$
$$
v_n = \prod_{k=1}^n \cosh\left(\frac{\theta}{2^k}\right) \Rightarrow 2^n \sinh\left(\frac{\theta}{2^n}\right)v_n = \sinh(\theta) \Rightarrow v_n = \frac{\sinh(\theta)}{2^n \sinh\left(\frac{\theta}{2^n}\right)} \to \frac{\sinh(\theta)}{\theta} 
$$

Donc $v_n \to \frac{\sinh(\theta)}{\theta}$ où $\theta = \operatorname{arccosh}(u_0)$.

\section*{Intégrales}

\subsection*{Exercice 1}

1.
Soit $x \in [a, b] \subset \R$. Alors, 

\[
\begin{aligned}
|f(x)|^2 
&= \left|\int_a^x f'(t)\,dt\right|^2 \\
&\le \left(\int_a^x |f'(t)|\,dt\right)^2 \\
&= \left(\int_a^x |f'(t)|\cdot 1\,dt\right)^2 \\
&\le \left(\int_a^x |f'(t)|^2\,dt\right)
   \left(\int_a^x 1^2\,dt\right)
   \quad \text{(Cauchy--Schwarz)} \\
&= (x-a)\int_a^x |f'(t)|^2\,dt \\
&\le (x-a)\int_a^b |f'(t)|^2\,dt .
\end{aligned}
\]

Donc, on obtient 
\[
\int_{a}^{b} |f(x)|^2\,dx \le \frac{(b-a)^2}{2}\int_a^b |f'(t)|^2\,dt .
\]
\\
2. On considère la fonction $g: [a,b] \to \R$ définie par $g(x) = \int_a^x |f'(t)|\,dt$. Alors $g'(x) = |f'(x)|$ et $g(a) = 0$. 

Or, $$|f(x)| = \left|\int_a^x f'(t)\,dt\right| \le \int_a^x |f'(t)|\,dt = g(x).$$

Ainsi,
$$
\left|\int_{a}^{b}f(x)f'(x)\,dx\right| \le \int_{a}^{b}|f(x)||f'(x)|\,dx \le \int_{a}^{b}g(x)g'(x)\,dx = \frac{g(b)^2}{2}
$$

Or,
\[
g(b)=\int_a^b |f'(t)|\,dt.
\]
Par Cauchy--Schwarz,
\[
\left(\int_a^b |f'(t)|\,dt\right)^2
=\left(\int_a^b |f'(t)|\cdot 1\,dt\right)^2
\le \int_a^b |f'(t)|^2\,dt \int_a^b 1^2\,dt
=(b-a)\int_a^b |f'(t)|^2\,dt.
\]

Ainsi,
\[
\int_a^b g(x)g'(x)\,dx
=\frac{g(b)^2}{2}
\le \frac{b-a}{2}\int_a^b |f'(t)|^2\,dt,
\]
et par conséquent
\[
\left|\int_a^b f(x)f'(x)\,dx\right|
\le \frac{b-a}{2}\int_a^b |f'(x)|^2\,dx.
\]

3. \\

(1) Par Cauchy-Schwarz, le cas d'égalité occure ssi $f'(x) = cte$. Comme $f(a) = 0$, on a $f(x) = c(x-a)$ pour un certain $c \in \R$.

En vérifiant, 

$$
\int_{a}^{b} |c|^2|x-a|^2 dx = |c|^2 \frac{(b-a)^3}{3} \hspace{1em}\text{et} \hspace{1em} \frac{(b-a)^2}{2} \int_{a}^{b} |c|^2 dx = |c|^2 \frac{(b-a)^3}{2}
$$

Pour l'égalité, $|c| = 0$:

Donc, on a l'égalité ssi $f \equiv 0$. \\

(2) Par Cauchy-Schwarz, le cas d'égalité occure ssi $|f'(x)| = cte$. Comme $f(a) = 0$, on a $f(x) = c(x-a)$ pour un certain $c \in \R$.

En vérifiant,
$$
\left|\int_{a}^{b} c^2(x-a) dx\right| = |c|^2 \frac{(b-a)^2}{2} \hspace{1em}\text{et} \hspace{1em} \frac{b-a}{2} \int_{a}^{b} |c|^2 dx = |c|^2 \frac{(b-a)^2}{2}
$$

Donc, on a l'égalité ssi $f(x) = c(x-a)$ pour un certain $c \in \R$.


\subsection*{Exercice 2}
 
Pour montrer que $f$ est bornée, on va montrer que $\lim_{x \to \infty} f(x) = 0$, puisque $f$ est continue sur $\R^+$. 

On peut utiliser le theorème precedente et, ainsi, on affirme que f est uniformément continue.

Or, puisque $f \in \mathcal{C}^1$ et $f' \in L^2(\R^+)$, on a que 

$$
\forall (x,y) \in \R_+^2, \hspace{1em} |f(x) - f(y)|^2 \leq \left|\int_{x}^{y} |f'(t)|^2 dt\right| \leq \left|\int_{x}^{\infty} |f'(t)|^2 dt\right|.\left|\int_{x}^{y} 1^2 dt\right| \hspace{1em} \text{(Par Cauchy-Schwarz)} 
$$

$$
= K^2|x-y|
$$

Ainsi, soit $\epsilon > 0$, il existe $\delta = \frac{\epsilon^2}{K^2} > 0$ tel que pour tout $(x,y) \in \R_+^2$ avec $|x-y| < \delta$, on a $|f(x) - f(y)| < \epsilon$.

Alors, $f$ est uniformément continue sur $\R^+$, et donc, $\lim_{x \to \infty} f(x) = 0$
$$
\Rightarrow f \qquad \text{est bornée sur } \R^+ \qquad \qed
$$

