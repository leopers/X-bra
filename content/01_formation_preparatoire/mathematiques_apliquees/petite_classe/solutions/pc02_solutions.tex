\subsection*{Execice 1.}

Soit $\varepsilon>0$. Choisissons $M>0$ tel que
\[
e^{-M}<\varepsilon.
\]

\medskip

\textbf{1) Convergence sur $[0,M]$ }

Pour $u\in[0,1)$, posons $g(u)=\log(1-u)$. On a
\[
g''(u)=-\frac{1}{(1-u)^2}.
\]
Par Taylor--Lagrange à l'ordre $1$ en $0$, pour tout $u\in[0,1)$ il existe $\theta\in(0,1)$ tel que
\[
\log(1-u)= -u+\frac{g''(\theta)}{2}u^2
= -u-\frac{u^2}{2(1-\theta)^2}.
\]
Donc
\[
\bigl|\log(1-u)+u\bigr|
=\frac{u^2}{2(1-\theta)^2}
\le \frac{u^2}{2}.
\]

Fixons $x\in[0,M]$ et supposons $n\ge 2M$. En posant $u=\frac{x}{n}$, on a $0\le u\le 1$, d'où
\[
\left|n\log\!\left(1-\frac{x}{n}\right)+x\right|
= n\bigl|\log(1-u)+u\bigr|
\le n\cdot \frac{u^2}{2}
= \frac{x^2}{2n}
\le \frac{M^2}{2n}.
\]

Maintenant, on applique le théorème des accroissements finis à $\exp$ entre
\[
a = n\log\!\left(1-\frac{x}{n}\right) \quad\text{et}\quad b=-x.
\]
Il existe $\xi$ entre $a$ et $b$ tel que
\[
\left|e^a-e^b\right| = e^{\xi}\,|a-b|.
\]
Or $x\in[0,n]$ implique $1-\frac{x}{n}\in[0,1]$, donc $a\le 0$, et bien sûr $b\le 0$,
donc $\xi\le 0$ et ainsi $e^\xi\le 1$. Par conséquent,
\[
\left|\left(1-\frac{x}{n}\right)^n-e^{-x}\right|
=\left|e^a-e^b\right|
\le |a-b|
=\left|n\log\!\left(1-\frac{x}{n}\right)+x\right|
\le \frac{M^2}{2n}.
\]
Donc
\[
\sup_{x\in[0,M]}|f_n(x)-e^{-x}|
\le \frac{M^2}{2nß}\xrightarrow[n\to\infty]{}0.
\]
Il existe donc $N_1$ tel que, pour $n\ge N_1$,
\[
\sup_{x\in[0,M]}|f_n(x)-e^{-x}|\le \varepsilon.
\]

\medskip

\textbf{2) Convergence sur $[M,n]$.}

On utilise l'inégalité $1-t\le e^{-t}$ pour $t\ge0$. Pour $x\in[0,n]$ :
\[
\left(1-\frac{x}{n}\right)^n\le e^{-x}.
\]
Donc, pour $x\in[M,n]$,
\[
0\le e^{-x}-\left(1-\frac{x}{n}\right)^n\le e^{-x}\le e^{-M}<\varepsilon,
\]
d'où
\[
\sup_{x\in[M,n]}|f_n(x)-e^{-x}|\le \varepsilon.
\]

\medskip

\textbf{3) Convergence sur $(n,+\infty)$.}

Si $x>n$, alors $f_n(x)=0$ et donc
\[
|f_n(x)-e^{-x}|=e^{-x}\le e^{-n}.
\]
Il existe $N_2$ tel que, pour $n\ge N_2$, $e^{-n}\le \varepsilon$, donc
\[
\sup_{x>n}|f_n(x)-e^{-x}|\le \varepsilon.
\]

\medskip

\textbf{Conclusion.}

Pour $n\ge N:=\max(N_1,N_2,2M)$,
\[
\sup_{x\ge 0}|f_n(x)-e^{-x}|
=
\max\!\left\{
\sup_{x\in[0,M]}|f_n(x)-e^{-x}|,
\sup_{x\in[M,n]}|f_n(x)-e^{-x}|,
\sup_{x>n}|f_n(x)-e^{-x}|
\right\}
\le \varepsilon.
\]
Ainsi $f_n\to e^{-x}$ uniformément sur $\mathbb{R}_+$.
\qed

\subsection*{Exercice 2.}

On affirme que $P_n \to f$ uniformément et, ainsi, que $f$ est aussi un polynôme.

Soit $\varepsilon>0$. Puisque  $P_n\to f$ uniformément, par le Critère de Cauchy, il existe $N_0$ tel que,

$$\forall m,n\ge N_0, \forall x\in\mathbb{R}\qquad |P_n(x)-P_m(x)|\le \varepsilon.$$

On prend $\varepsilon = 1$ et donc $$\forall m,n\ge N_0, \forall x\in\mathbb{R}\qquad |P_n(x)-P_m(x)|\le 1.$$

Cela est satisfait ssi $P_n(x)-P_m(x)= \text{cte}$, puisque $P_n-P_m$ est un polynôme et on sait que les polynômes sont non bornés à l'infini sauf s'ils sont constants. 

Ainsi, 

$$\forall n\ge N_0, \forall x\in\mathbb{R}\qquad P_n(x)-P_{N_0}(x)= \alpha_n $$

où $\alpha_n$ est une constante réel qui dépend de $n$.

Mais, $$\alpha_n = P_n(0)-P_{N_0}(0) \xrightarrow[n\to\infty]{} f(0)-P_{N_0}(0) = \alpha,$$

 où $\alpha \in \mathbb{R}$ puisque $P_n(0)\to f(0)$ par convergence uniforme.

Donc, pour $n\ge N_0$ et $x\in\mathbb{R}$, $$P_n(x) = P_{N_0}(x) + \alpha_n \xrightarrow[n\to\infty]{} P_{N_0}(x) + \alpha.$$

Ainsi, $f(x) = P_{N_0}(x) + \alpha$ est un polynôme.
\qed

\subsection*{Exercice 3.}

