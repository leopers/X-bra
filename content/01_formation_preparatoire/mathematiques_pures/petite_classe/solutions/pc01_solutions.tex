% =========================================================
\section*{Algèbre Linéaire}
\addcontentsline{toc}{section}{Algèbre Linéaire}
% =========================================================

\subsection*{Exercice 1}

Comme $\ker f$ est un sous-espace vectoriel de $E$, on peut choisir une base de $\ker f$: $B_{\ker} = (v_1, v_2, \cdots, v_k)$. Complétons la base, avec des vecteurs $u_i$ afin de construire une base de $E$:

\[B_E = (v_1, v_2, \cdots, v_k, u_{k+1}, \cdots, u_n)\]

Maintenant, voyez que pour chaque vecteur $w$ appartenant à l'image de $f$, on peut l'écrire comme:

\[w = f(v_E) = f\left(\lambda_1 v_1 + \lambda_2 v_2 + \cdots + \lambda_k v_k + \lambda_{k+1} u_{k+1} + \cdots + \lambda_n u_n\right)\]

Mais, $f(v_i) = 0,\ i \le k$, alors:

\[\begin{split}
	w &= f\left(\lambda_{k+1} u_{k+1} + \cdots + \lambda_n u_n\right) \\
	 &= \lambda_{k+1} f(u_{k+1}) + \cdots + \lambda_nf(u_n)
	\end{split}\]

Donc on voie qu'ils sont combinaisons linéaires du ensemble $A = (f(u_{k+1}), \cdots, f(u_n)$ (on dit que $A$ engendre $\imag f$). Mais $A$ est aussi linéairement indépendant, car $w = 0$ si et seulement si:

\[\lambda_{k+1} u_{k+1} + \cdots + \lambda_n u_n = v' \in \ker f\]

Mais comme $v' \in \ker f$ donc $v' = \alpha_1 v_1 + \cdots + \alpha_kv_k$, donc:

\[-\alpha_1 v_1 - \cdots - \alpha_k v_k + \lambda_{k+1} u_{k+1} + \cdots + \lambda_n u_n = 0\]

Mais, comme $B_E$ est une base de $E$, cela n'arrive que si $\alpha_1 = \cdots = \alpha_k = \lambda_{k+1} = \cdots = \lambda_n = 0$.

Alors $A$ est une base de $\imag f$, donc:

\[\dim \ker f + \dim A = \dim E\]

\[\dim \ker f + \dim \imag f = \dim E\ \QED\]



\subsection*{Exercice 2}

On doit demontrer que $1 \implies 2 \implies 3 \implies 1$ (ou quelque autre combinaison cyclique):


\begin{enumerate}
	\item $1 \implies 2$
	
	C'est direct que $\imag f^2 \subset \imag f$, donc on doit montrer que $\imag f \subset \imag f^2$.
	
	Soit $v \in \imag f$ on a que: $v = f(u)$. Comme l'espace $V = \ker f \oplus \imag f$, on peut dire que $u = x + f(w)$ où $x \in \ker f $ et $w \notin \ker f$. $v = f(u) = f(x + f(w)) = f(f(w)) \in \imag f^2$. Donc $\imag f \subset \imag f^2\ \QED$.
	
	
	\item $2 \implies 3 $
	
	C'est direct que $\ker f \subset \ker f^2$.
	
	Pour le théorème du rang (démontré dans l'exercice 1):
	
	\[\begin{cases}
		\dim \imag f + \dim \ker f = n \\
		\dim \imag f^2 + \dim \ker f^2 = n 
	\end{cases}\]
	
	Mais comme $\dim \imag f^2 = \dim \imag f$ alors $\dim \ker f = \dim \ker f^2$, donc:
	
	\[\ker f = \ker f^2\ \QED\] 
	
	
	\item $3 \implies 1$
	
	Si l'on montre que $\ker (f) \cap \imag f = \{0\}$, par le théorème du rang, on montre aussi que $\ker f + \imag f = \ker(f) \oplus \imag f = V$.
	
	En supposant $v \in \ker f$ et $v \in \imag f$, on a que
	
	
	\[\begin{cases}
		f(v) = 0 \\
		v = f(u)\ |\ $u \in V$
	\end{cases}\].
	
	\[\implies f(f(u)) = 0\]
	
	Alors on voit que $u \in \ker f^2 = \ker f\ \implies f(u) = 0$, donc $v = 0$. Ainsi $\ker (f) \cap \imag f = \{0\}$ $\blacksquare$.
	
\end{enumerate}



\subsection*{Exercice 3}


Par le théorème du rang: $u_n = v_n$, de façon qu'on ne doit analyser qu'une suite.

L'intuition 



