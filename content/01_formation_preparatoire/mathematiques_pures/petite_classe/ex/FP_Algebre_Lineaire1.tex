\subsection*{Feuille n°1 — Algèbre linéaire}

\textbf{Exercice 1.} Soit $E$ et $F$ deux espaces vectoriels de dimension finie. Soit $f$ une application linéaire de $E$ dans $F$. Montrer que
\[ \dim \imag f + \dim \ker f = \dim E. \]

\textbf{Exercice 2.} Soit $V$ un espace vectoriel de dimension $n$ et $f \in \End(V)$. Montrer que les assertions suivantes sont équivalentes :
\begin{enumerate}
    \item $\ker(f) \oplus \imag(f) = V$ ;
    \item $\imag(f) = \imag(f^2)$ ;
    \item $\ker(f) = \ker(f^2)$.
\end{enumerate}

\textbf{Exercice 3.} Soit $E$ un espace vectoriel de dimension finie et $f \in \mathcal{L}(E)$. Montrer que les suites $(u_n)$ et $(v_n)$ définies par
\[ u_n = \dim (\imag f^n) - \dim (\imag f^{n+1}), \]
\[ v_n = \dim (\ker f^{n+1}) - \dim (\ker f^n) \]
sont des suites positives, décroissantes qui tendent vers $0$.

\textbf{Exercice 4.} Soit $\C_n[X]$ l'espace vectoriel des polynômes à coefficients dans $\C$ de degré inférieur ou égal à $n$. On considère l'application linéaire $\varphi$ qui à $P(X) \in \C_n[X]$ associe $P(X+1) - P(X)$.

Déterminer son noyau $\ker(\varphi)$ et son image $\imag(\varphi)$.

\textbf{Exercice 5.} Soient $f$ et $g$ deux applications linéaires de $\R^n$ dans $\R$. Montrer que
\begin{center}
	(a) $\ker(f) = \ker(g)$ (égalité des noyaux)
\end{center}


si et seulement si

\begin{center}
    (b) il existe $\lambda \in \R, \lambda \neq 0$ tel que $f = \lambda g$.
\end{center}


\textbf{Exercice 6.} Soit $m \in \R$ et $A(m) = \begin{pmatrix} -1 & 0 & 0 \\ 2 & m & -2 \\ 2 - m & m - 2 & m \end{pmatrix}$
\begin{enumerate}
    \item Déterminer le polynôme caractéristique et le polynôme minimal de $A(m)$.
    \item Pour quelles valeurs de $m$ la matrice $A(m)$ est-elle diagonalisable ?
    \item Lorsque $A(m)$ est diagonalisable, trouver une matrice $P$ inversible et une matrice diagonale $D$ telles que $A(m) = PDP^{-1}$.
\end{enumerate}

\textbf{Exercice 7.} Montrer que la matrice $M(a) = \begin{pmatrix} 2 & 0 & 1 & -a \\ 1 & 1 & 0 & 1 \\ a & -1 & 0 & 1 \\ -1 & 0 & 1 & 2a \end{pmatrix}$ est diagonalisable si et seulement si $a = 1$. Dans ce cas trouver $P$ tel que $P^{-1} M(1) P$ soit une matrice diagonale que l'on déterminera.

\textbf{Exercice 8.} Soit $E=\R_n[X]$ l’espace vectoriel des polynômes de degré inférieur ou égal à $n$.  
Soient $a_0,a_1,\dots,a_n$ des nombres réels deux à deux distincts.  
Pour $j\in\{0,\dots,n\}$, on note $\varphi_j\in E^{*}$ défini par $\varphi_j(P)=P(a_j)$.

\begin{enumerate}
	\item Montrer que $(\varphi_0,\dots,\varphi_n)$ est une base de $E^{*}$.
	\item Trouver la base $(P_0,P_1,\dots,P_n)$ de $E$ dont la base duale est $(\varphi_0,\dots,\varphi_n)$.
\end{enumerate}

\textbf{Exercice 9.} Soit $V$ un espace vectoriel de dimension $d$ sur $\C$ et $f$ un endomorphisme de $V$. On note $C_f$ son polynôme caractéristique.

I. Soit $v$ un vecteur non nul de $V$. Soit $n$ le plus grand entier tel que la famille $(v,f(v),\dots,f^{n-1}(v))$ soit libre.  
On note $W$ le sous-espace vectoriel engendré par cette famille et on pose
\[
f^{n}(v)=a_0v+a_1f(v)+\dots+a_{n-1}f^{n-1}(v)
\]

\begin{enumerate}
	\item Montrer que $W$ est stable par $f$ (c’est-à-dire $f(W)\subset W$). On note alors $f_W$ la restriction de $f$ à $W$.
	\item Quel est le polynôme caractéristique $C_{f_W}$ de $f_W$ ?
	\item Montrer que $C_{f_W}(f_W)=0$.
	\item Montrer que $C_{f_W}$ divise $C_f$.
\end{enumerate}

II. Déduire de la première partie que $C_f(f)=0$ (Théorème de Cayley-Hamilton).

\textbf{Exercice 10.} Soit $V$ un espace vectoriel de dimension $n$.  
Soient $f$ et $g$ deux endomorphismes diagonalisables de $V$ qui commutent (c’est‑à‑dire $f\circ g=g\circ f$).

\begin{enumerate}
	\item Soit $E$ un sous‑espace propre de $f$. Montrer que $E$ est stable par $g$ (c’est‑à‑dire $g(E)\subseteq E$).
	\item Montrer que la restriction de $g$ à $E$ est un endomorphisme diagonalisable de $E$.
	\item Montrer que $f$ et $g$ sont diagonalisables dans une même base.
	\item \emph{(difficile)} Soit $(f_i)_{i\in I}$ une famille d’endomorphismes diagonalisables de $V$. On suppose que les $f_i$ commutent deux à deux.  
	Montrer par récurrence sur $\dim(V)$ que les $f_i$, pour $i\in I$, sont diagonalisables dans une même base.
\end{enumerate}

\textbf{Exercice 11.} Soit $G$ un sous‑groupe fini de $\mathrm{GL}(n,\mathbf{R})$.  
On suppose que tout élément $A$ de $G$ satisfait $A^{2}=I_n$ où $I_n$ est la matrice identité.  
Montrer que $G$ a au plus $2^{\,n}$ éléments.

\textbf{Exercice 12.} Soit $V$ un espace vectoriel de dimension $n$ et $f$ un endomorphisme de $V$.

\begin{enumerate}
	\item Montrer que $f$ est nilpotent (c’est‑à‑dire qu’il existe $r\in\mathbf{N}^{*}$ tel que $f^{r}=0$) si et seulement si son polynôme caractéristique est $P(T)=(-T)^{n}$.
	\item On suppose que $f$ est nilpotent et $f^{\,n-1}\neq 0$.
	\begin{enumerate}
		\item Montrer qu’il existe $v\in V$ tel que $v,\;f(v),\;f^{2}(v),\dots,f^{\,n-1}(v)$ soit une base de $V$.
		\item Écrire la matrice de $f$ dans la base $(v,f(v),f^{2}(v),\dots,f^{\,n-1}(v))$. En déduire le rang de $f$.
		\item Montrer que $g$ commute à $f$ si et seulement si il existe un polynôme $P\in\mathbf{R}[X]$ tel que $g=P(f)$.
	\end{enumerate}
\end{enumerate}

