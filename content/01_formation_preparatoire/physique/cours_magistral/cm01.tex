% =========================================================
\section*{Oscillateur harmonique classique}
\addcontentsline{toc}{section}{Oscillateur harmonique classique}
% =========================================================

On considère un oscillateur harmonique classique de masse $m$ et de
pulsation $\omega$, décrit par le Hamiltonien
\[
H(x,p)=\frac{p^2}{2m}+\frac{1}{2}m\omega^2 x^2.
\]

Pour une énergie fixée $E$, le mouvement est périodique et entièrement
déterminé par la condition
\[
H(x,p)=E.
\]

% =========================================================
\subsection*{Espace des phases et trajectoire}
\addcontentsline{toc}{subsection}{Espace des phases et trajectoire}
% =========================================================

Dans l’espace des phases $(x,p)$, l’équation $H(x,p)=E$ définit une
ellipse.
Les demi-axes de cette ellipse sont
\[
x_{\max}=\sqrt{\frac{2E}{m\omega^2}},
\qquad
p_{\max}=\sqrt{2mE}.
\]

\begin{figure}[h]
\centering
\begin{tikzpicture}[x=1cm,y=1cm,>=Latex]

  % --- Visual semi-axes (in cm because x=1cm, y=1cm) ---
  \def\xa{3.6}
  \def\pb{2.2}

  % --- Axes ---
  \draw[->] (-4.4,0) -- (4.6,0) node[below right] {$x$};
  \draw[->] (0,-3.0) -- (0,3.1) node[above left] {$p$};

  % --- Shaded area of the ellipse ---
  \fill[XBlue,opacity=0.10]
    plot[domain=0:360, samples=240]
    ({\xa*cos(\x)}, {\pb*sin(\x)}) -- cycle;

  % --- Ellipse boundary ---
  \draw[XBlue,line width=1.0pt]
    plot[domain=0:360, samples=240]
    ({\xa*cos(\x)}, {\pb*sin(\x)}) -- cycle;

  % --- Direction arrow on the orbit (time evolution) ---
  \draw[XBlue,->,line width=0.9pt]
    plot[domain=20:55, samples=40]
    ({\xa*cos(\x)}, {\pb*sin(\x)});

  % --- Extremal points ---
  \fill ( \xa,0) circle (1.2pt);
  \fill (-\xa,0) circle (1.2pt);
  \fill (0, \pb) circle (1.2pt);
  \fill (0,-\pb) circle (1.2pt);

  % --- Labels: x_max and p_max ---
  \draw[<->] (0,-0.55) -- (\xa,-0.55);
  \node[below] at ({\xa/2},-0.55) {$x_{\max}$};

  \draw[<->] (0.55,0) -- (0.55,\pb);
  \node[right] at (0.55,{\pb/2}) {$p_{\max}$};

  % --- Area label ---
  \node at (1.1,0.8) {$\mathcal{A}$};

  % --- Energy shell label ---
  \node[align=left] at (1.9,2.45) {$H(x,p)=E$};

\end{tikzpicture}
\caption{Trajectoire en espace des phases $(x,p)$ pour l’oscillateur harmonique à énergie fixée : une ellipse. L’aire $\mathcal{A}$ intervient dans l’action $J=\frac{1}{2\pi}\oint p\,\dd x$.}
\end{figure}


Cette représentation géométrique sera essentielle pour introduire
la notion d’action.

% =========================================================
\subsection*{Action et interprétation géométrique}
\addcontentsline{toc}{subsection}{Action et interprétation géométrique}
% =========================================================

On définit l’action $J$ associée à un mouvement périodique par
\[
J=\frac{1}{2\pi}\oint p\,\dd x.
\]

\begin{theorem}
L'action $J$ est égale
à l’aire de l’ellipse dans l’espace des phases divisée par $2\pi$,
et vérifie
\[
J=\frac{E}{\omega}.
\]
\end{theorem}

\begin{proof}
L’aire $\mathcal{A}$ de l’ellipse de demi-axes $x_{\max}$ et $p_{\max}$
est donnée par
\[
\mathcal{A}=\pi x_{\max} p_{\max}
= \pi \sqrt{\frac{2E}{m\omega^2}} \sqrt{2mE}
= \frac{2\pi E}{\omega}.
\]
Par définition de l’action, on a donc $J=\mathcal{A}/(2\pi)=E/\omega$.
\end{proof}

% =========================================================
\section*{Interférences de Young}
\addcontentsline{toc}{section}{Interférences de Young}
% =========================================================

On considère une source monochromatique de pulsation $\omega$ et de longueur d’onde
$\lambda$ (nombre d’onde $k = 2\pi/\lambda$), éclairant deux fentes $S_1$ et $S_2$
séparées d’une distance $a$. Un écran est placé à une distance $D \gg a$.
On observe le champ en un point $M$ de l’écran, repéré par sa coordonnée transverse $y$
(par rapport à l’axe central).

\begin{figure}[h]
\centering
\begin{tikzpicture}[x=1cm,y=1cm,>=Latex,scale=1.0]

  % Parameters
  \def\a{2.2}      % slit separation (in cm, drawn)
  \def\D{8.0}      % screen distance
  \def\yM{2.2}     % observation point height

  % Axes / baseline (optional)
  %\draw[->,black!40] (-0.5,0) -- (\D+1,0) node[below] {$x$};

  % Double-slit plane
  \draw[black!70,line width=0.7pt] (0,-3.2) -- (0,3.2);
  \node[black!70,rotate=90] at (-0.25,0) {Fentes};

  % Slits positions
  \coordinate (S1) at (0, \a/2);
  \coordinate (S2) at (0,-\a/2);

  % Draw slits as small openings
  \draw[line width=2.0pt,white] (0, \a/2-0.18) -- (0, \a/2+0.18);
  \draw[line width=2.0pt,white] (0,-\a/2-0.18) -- (0,-\a/2+0.18);
  \fill[black] (S1) circle (0.8pt);
  \fill[black] (S2) circle (0.8pt);
  \node[left] at (S1) {$S_1$};
  \node[left] at (S2) {$S_2$};

  % Screen
  \draw[black!70,line width=0.7pt] (\D,-3.2) -- (\D,3.2);
  \node[black!70,rotate=90] at (\D+0.25,0) {Écran};

  % Observation point M
  \coordinate (M) at (\D,\yM);
  \fill[black] (M) circle (1.2pt);
  \node[right] at (M) {$M$};

  % Central point O on screen
  \coordinate (O) at (\D,0);
  \fill[black] (O) circle (0.8pt);
  \node[right] at (O) {$O$};

  % Rays
  \draw[XBlue,line width=1.0pt] (S1) -- (M);
  \draw[XBlue,line width=1.0pt] (S2) -- (M);

  % Indicate distances r1, r2 along rays (small labels)
  \node[fill=white,inner sep=1pt] at (4.2,2.1) {$r_1$};
  \node[fill=white,inner sep=1pt] at (4.4,0.9) {$r_2$};

  % Mark y on screen
  \draw[<->,black!70] (\D+0.7,0) -- (\D+0.7,\yM);
  \node[right,black!70] at (\D+0.7,\yM/2) {$y$};

  % Baseline for theta from slit midpoint to M
  \coordinate (C) at (0,0); % midpoint between slits
  \draw[black!40,dashed] (C) -- (\D,0);
  \draw[black!40,dashed] (C) -- (M);

  % Angle theta at C
  \draw[black!60] (1.1,0) arc (0:{atan(\yM/\D)}:1.1);
  \node[black!60] at (1.35,0.35) {$\theta$};

  % Indicate slit separation a
  \draw[<->,black!70] (-0.6,-\a/2) -- (-0.6,\a/2);
  \node[left,black!70] at (-0.6,0) {$a$};

  % Far-field note
  \node[black!50] at (4.0,-2.6) {$D \gg a \;\;(\text{approx. champ lointain})$};

\end{tikzpicture}
\caption{Montage de Young : deux fentes $S_1,S_2$ séparées de $a$, écran à distance $D$,
observation en $M$ (coordonnée $y$). La différence de marche vaut $\delta = r_1-r_2 \simeq a\sin\theta \simeq a\,y/D$.}
\end{figure}

\paragraph{Champ complexe en $M$.}
On modélise l’onde issue de chaque fente par une onde cylindrique.
Le champ complexe en $M$ s’écrit
\[
\mathcal{E}(M,t)=\mathcal{E}_1(M,t)+\mathcal{E}_2(M,t),
\qquad
\mathcal{E}_j(M,t)=E_{0j}\,e^{i(\omega t-k r_j+\varphi_j)}.
\]
Ici $r_j=|S_jM|$ et $\varphi_j$ est une phase initiale (liée à la source et au trajet amont).

En champ lointain, on a $r_1\simeq r_2\simeq D$ pour les amplitudes, donc on peut prendre
$E_{01}\simeq E_{02}=E_0$ (mêmes intensités issues des fentes). La dépendance en phase est elle
cruciale :
\[
\Delta\phi = ( -kr_1+\varphi_1)-(-kr_2+\varphi_2)= -k(r_1-r_2)+(\varphi_1-\varphi_2)
= -k\delta + \Delta\varphi.
\]

\paragraph{Différence de marche.}
Géométriquement,
\[
\delta=r_1-r_2 \simeq a\sin\theta \simeq a\,\frac{y}{D}
\quad (D\gg a,\; \theta \text{ petit}).
\]
Si l’illumination des deux fentes est cohérente (même source), on peut prendre $\Delta\varphi=0$
(ou constante), ce qui donne
\[
\Delta\phi \simeq -k\,a\frac{y}{D}.
\]

\paragraph{Intensité observée.}
L’intensité est proportionnelle à la moyenne temporelle de $|\mathcal{E}|^2$ :
\[
I(M)\propto \langle |\mathcal{E}_1+\mathcal{E}_2|^2\rangle
= \langle |\mathcal{E}_1|^2\rangle + \langle |\mathcal{E}_2|^2\rangle
+ 2\Re\langle \mathcal{E}_1\mathcal{E}_2^\ast\rangle.
\]
En supposant $E_{01}=E_{02}=E_0$ et cohérence parfaite, on obtient
\[
I(M)=I_1+I_2+2\sqrt{I_1 I_2}\cos(\Delta\phi).
\]
En particulier si $I_1=I_2=I_0$, alors
\[
I(M)=2I_0\bigl(1+\cos(\Delta\phi)\bigr)=4I_0\cos^2\!\Bigl(\frac{\Delta\phi}{2}\Bigr).
\]
Avec $\Delta\phi\simeq k\,a\,y/D$ (au signe près), on obtient les franges :
\[
I(y)=4I_0\cos^2\!\Bigl(\frac{\pi a}{\lambda D}\,y\Bigr).
\]

\paragraph{Positions des franges.}
Les maxima (constructif) satisfont $\Delta\phi =2\pi m$ :
\[
y_m \simeq m\,\frac{\lambda D}{a}, \qquad m\in\mathbb{Z},
\]
et l’interfrange vaut
\[
i = y_{m+1}-y_m=\frac{\lambda D}{a}.
\]

\begin{intuitionbox}
L’interférence est un phénomène \emph{d’addition des amplitudes complexes} :
les intensités ne s’additionnent pas directement, c’est le terme croisé
$2\Re(\mathcal{E}_1\mathcal{E}_2^\ast)$ qui crée les franges via $\cos(\Delta\phi)$.
\end{intuitionbox}

% =========================================================
\section*{Équations d’onde pour $\mathbf{E}$ et $\mathbf{B}$ à partir de Maxwell}
\addcontentsline{toc}{section}{Équations d’onde pour $\mathbf{E}$ et $\mathbf{B}$ à partir de Maxwell}
% =========================================================

On se place dans le vide (ou plus généralement dans un milieu homogène, linéaire, isotrope),
sans charges ni courants libres :
\[
\rho = 0,\qquad \mathbf{j}=\mathbf{0}.
\]
Les équations de Maxwell s’écrivent alors
\[
\nabla\cdot\mathbf{E}=0,\qquad
\nabla\cdot\mathbf{B}=0,\qquad
\nabla\times\mathbf{E}=-\frac{\partial \mathbf{B}}{\partial t},\qquad
\nabla\times\mathbf{B}=\mu_0\varepsilon_0\,\frac{\partial \mathbf{E}}{\partial t}.
\]
On note $c=\dfrac{1}{\sqrt{\mu_0\varepsilon_0}}$ la célérité des ondes électromagnétiques dans le vide.

% =========================================================
\subsection*{Équation d’onde pour le champ électrique}
\addcontentsline{toc}{subsection}{Équation d’onde pour le champ électrique}
% =========================================================

On applique l’opérateur rotationnel $\nabla\times$ à l’équation de Faraday :
\[
\nabla\times(\nabla\times\mathbf{E})
= -\frac{\partial}{\partial t}\bigl(\nabla\times\mathbf{B}\bigr).
\]
En utilisant l’équation d’Ampère-Maxwell dans le vide,
\[
\nabla\times\mathbf{B}=\mu_0\varepsilon_0\,\frac{\partial \mathbf{E}}{\partial t},
\]
on obtient
\[
\nabla\times(\nabla\times\mathbf{E})
= -\mu_0\varepsilon_0\,\frac{\partial^2 \mathbf{E}}{\partial t^2}.
\]
On utilise ensuite l’identité vectorielle (justifiée dans la remarque ci-dessous)
\[
\nabla\times(\nabla\times\mathbf{E})
=\nabla(\nabla\cdot\mathbf{E})-\nabla^2\mathbf{E}.
\]
Or dans le vide $\nabla\cdot\mathbf{E}=0$, donc
\[
-\nabla^2\mathbf{E}=-\mu_0\varepsilon_0\,\frac{\partial^2 \mathbf{E}}{\partial t^2},
\]
c’est-à-dire l’équation d’onde vectorielle
\[
\boxed{\;\nabla^2\mathbf{E}-\frac{1}{c^2}\,\frac{\partial^2 \mathbf{E}}{\partial t^2}= \mathbf{0}\;}
\qquad\text{avec}\qquad
c=\frac{1}{\sqrt{\mu_0\varepsilon_0}}.
\]

% =========================================================
\subsection*{Équation d’onde pour le champ magnétique}
\addcontentsline{toc}{subsection}{Équation d’onde pour le champ magnétique}
% =========================================================

On procède de manière analogue. On applique $\nabla\times$ à l’équation d’Ampère-Maxwell :
\[
\nabla\times(\nabla\times\mathbf{B})
=\mu_0\varepsilon_0\,\frac{\partial}{\partial t}\bigl(\nabla\times\mathbf{E}\bigr).
\]
En utilisant Faraday $\nabla\times\mathbf{E}=-\dfrac{\partial \mathbf{B}}{\partial t}$, on trouve
\[
\nabla\times(\nabla\times\mathbf{B})
=-\mu_0\varepsilon_0\,\frac{\partial^2\mathbf{B}}{\partial t^2}.
\]
Puis, grâce à l’identité
\[
\nabla\times(\nabla\times\mathbf{B})
=\nabla(\nabla\cdot\mathbf{B})-\nabla^2\mathbf{B},
\]
et au fait que $\nabla\cdot\mathbf{B}=0$, on obtient finalement
\[
\boxed{\;\nabla^2\mathbf{B}-\frac{1}{c^2}\,\frac{\partial^2 \mathbf{B}}{\partial t^2}= \mathbf{0}\;}
\qquad\text{avec}\qquad
c=\frac{1}{\sqrt{\mu_0\varepsilon_0}}.
\]

\begin{intuitionbox}
Dans le vide, $\mathbf{E}$ et $\mathbf{B}$ sont couplés par Maxwell : une variation temporelle de $\mathbf{B}$
crée du rotationnel de $\mathbf{E}$ (Faraday), et une variation temporelle de $\mathbf{E}$ crée du rotationnel de $\mathbf{B}$
(Ampère-Maxwell). Cette boucle dynamique engendre une propagation : l’onde électromagnétique.
\end{intuitionbox}

% =========================================================
\subsection*{Remarque : identité du double rotationnel}
% =========================================================

\begin{remarkbox}
\textbf{Énoncé.} Pour tout champ vectoriel $\mathbf{A}$ de classe $\mathcal{C}^2$ sur un ouvert de $\mathbb{R}^3$,
on a l’identité fondamentale
\[
\nabla\times(\nabla\times\mathbf{A})=\nabla(\nabla\cdot\mathbf{A})-\nabla^2\mathbf{A},
\]
où $\nabla^2\mathbf{A}$ désigne le Laplacien appliqué composante par composante :
$\nabla^2\mathbf{A}=(\nabla^2A_x,\nabla^2A_y,\nabla^2A_z)$.

\medskip
\textbf{Démonstration.}
Écrivons $\mathbf{A}=(A_x,A_y,A_z)$.
Par définition du rotationnel,
\[
\nabla\times\mathbf{A}=
\bigl(\partial_yA_z-\partial_zA_y,\ \partial_zA_x-\partial_xA_z,\ \partial_xA_y-\partial_yA_x\bigr).
\]
On calcule ensuite $\nabla\times(\nabla\times\mathbf{A})$.
On commence par sa première composante :
\[
\bigl(\nabla\times(\nabla\times\mathbf{A})\bigr)_x
=\partial_y\bigl((\nabla\times\mathbf{A})_z\bigr)-\partial_z\bigl((\nabla\times\mathbf{A})_y\bigr).
\]
En remplaçant $(\nabla\times\mathbf{A})_z=\partial_xA_y-\partial_yA_x$ et
$(\nabla\times\mathbf{A})_y=\partial_zA_x-\partial_xA_z$, on obtient
\[
\bigl(\nabla\times(\nabla\times\mathbf{A})\bigr)_x
=\partial_y(\partial_xA_y-\partial_yA_x)-\partial_z(\partial_zA_x-\partial_xA_z).
\]
Donc, en développant et en permutant les dérivées mixtes (hypothèse $\mathcal{C}^2$),
\[
\bigl(\nabla\times(\nabla\times\mathbf{A})\bigr)_x
=\partial_x(\partial_yA_y+\partial_zA_z)-(\partial_{yy}A_x+\partial_{zz}A_x).
\]
Or $\partial_yA_y+\partial_zA_z = (\nabla\cdot\mathbf{A})-\partial_xA_x$, donc
\[
\bigl(\nabla\times(\nabla\times\mathbf{A})\bigr)_x
=\partial_x(\nabla\cdot\mathbf{A})-\nabla^2A_x.
\]
On procède de façon identique pour les composantes $y$ et $z$ (même structure de calcul),
ce qui donne au final
\[
\nabla\times(\nabla\times\mathbf{A})
=\bigl(\partial_x(\nabla\cdot\mathbf{A})-\nabla^2A_x,\ 
       \partial_y(\nabla\cdot\mathbf{A})-\nabla^2A_y,\ 
       \partial_z(\nabla\cdot\mathbf{A})-\nabla^2A_z\bigr),
\]
c’est-à-dire bien
\[
\nabla\times(\nabla\times\mathbf{A})=\nabla(\nabla\cdot\mathbf{A})-\nabla^2\mathbf{A}.
\]

\end{remarkbox}

% =========================================================
\subsection*{Solution générale de l’équation d’onde 1D}
\addcontentsline{toc}{subsection}{Solution générale de l’équation d’onde 1D}
% =========================================================

On considère l’équation d’onde unidimensionnelle
\[
\frac{\partial^2 u}{\partial t^2}(x,t)=c^2\,\frac{\partial^2 u}{\partial x^2}(x,t),
\qquad (x,t)\in\mathbb{R}\times\mathbb{R}.
\]
On cherche la forme générale des solutions suffisamment régulières (au moins $\mathcal{C}^2$).

\paragraph{Changement de variables.}
On introduit les variables caractéristiques (sans utiliser la méthode des caractéristiques comme outil conceptuel,
simplement comme un changement de coordonnées) :
\[
\xi=x-ct,\qquad \eta=x+ct.
\]
On peut inverser :
\[
x=\frac{\xi+\eta}{2},\qquad t=\frac{\eta-\xi}{2c}.
\]
On pose alors $U(\xi,\eta)=u(x,t)=u\!\left(\frac{\xi+\eta}{2},\,\frac{\eta-\xi}{2c}\right)$.

\paragraph{Règles de dérivation.}
Par la règle de la chaîne,
\[
\frac{\partial}{\partial x}
=\frac{\partial \xi}{\partial x}\frac{\partial}{\partial \xi}
+\frac{\partial \eta}{\partial x}\frac{\partial}{\partial \eta}
=\frac{\partial}{\partial \xi}+\frac{\partial}{\partial \eta},
\]
et
\[
\frac{\partial}{\partial t}
=\frac{\partial \xi}{\partial t}\frac{\partial}{\partial \xi}
+\frac{\partial \eta}{\partial t}\frac{\partial}{\partial \eta}
=-c\,\frac{\partial}{\partial \xi}+c\,\frac{\partial}{\partial \eta}
=c\!\left(\frac{\partial}{\partial \eta}-\frac{\partial}{\partial \xi}\right).
\]

On en déduit les dérivées secondes.
D’abord,
\[
\frac{\partial^2}{\partial x^2}
=\left(\frac{\partial}{\partial \xi}+\frac{\partial}{\partial \eta}\right)^2
=\frac{\partial^2}{\partial \xi^2}+2\frac{\partial^2}{\partial \xi\partial \eta}+\frac{\partial^2}{\partial \eta^2}.
\]
Ensuite,
\[
\frac{\partial^2}{\partial t^2}
=c^2\left(\frac{\partial}{\partial \eta}-\frac{\partial}{\partial \xi}\right)^2
=c^2\left(\frac{\partial^2}{\partial \eta^2}-2\frac{\partial^2}{\partial \xi\partial \eta}+\frac{\partial^2}{\partial \xi^2}\right).
\]

\paragraph{Réduction de l’équation.}
En substituant dans l’équation d’onde $u_{tt}=c^2u_{xx}$, on obtient
\[
c^2\left(U_{\eta\eta}-2U_{\xi\eta}+U_{\xi\xi}\right)
=c^2\left(U_{\xi\xi}+2U_{\xi\eta}+U_{\eta\eta}\right).
\]
On simplifie par $c^2$ puis par $U_{\xi\xi}+U_{\eta\eta}$ des deux côtés :
\[
-2U_{\xi\eta}=+2U_{\xi\eta}
\qquad\Longrightarrow\qquad
U_{\xi\eta}=0.
\]

\paragraph{Intégration.}
L’équation $U_{\xi\eta}=0$ signifie que $\partial_\eta(U_\xi)=0$, donc $U_\xi$ ne dépend pas de $\eta$.
Il existe donc une fonction $F$ telle que
\[
U_\xi(\xi,\eta)=F(\xi).
\]
En intégrant par rapport à $\xi$,
\[
U(\xi,\eta)=\int F(\xi)\,\dd\xi + G(\eta)=f(\xi)+g(\eta),
\]
où $f$ est une primitive de $F$ et $g$ une fonction arbitraire de $\eta$.

En revenant aux variables $(x,t)$, on obtient la forme générale :
\[
\boxed{\;u(x,t)=f(x-ct)+g(x+ct)\;}
\]
où $f$ et $g$ sont deux fonctions arbitraires (de classe $\mathcal{C}^2$ si l’on veut une solution classique).

\begin{intuitionbox}
Le terme $f(x-ct)$ représente une onde se propageant vers la droite à la vitesse $c$,
tandis que $g(x+ct)$ se propage vers la gauche. La solution générale est la superposition
de ces deux ondes progressives.
\end{intuitionbox}

% =========================================================
\section*{Ondes : onde plane harmonique, relation de dispersion, vitesses de phase et de groupe}
\addcontentsline{toc}{section}{Ondes : OPH, dispersion, vitesses de phase et de groupe}
% =========================================================

On approfondit ici le formalisme mathématique des ondes, en partant de l’équation d’onde
et en introduisant systématiquement les solutions harmoniques, la relation de dispersion
et la construction des paquets d’ondes (méthode de phase stationnaire / développement local).

% =========================================================
\subsection*{Équation d’onde et ansatz harmonique}
\addcontentsline{toc}{subsection}{Équation d’onde et ansatz harmonique}
% =========================================================

En dimension 1, l’équation d’onde s’écrit
\[
\partial_{tt}u=c^2\,\partial_{xx}u.
\]
On cherche une solution complexe (on prendra la partie réelle à la fin) sous la forme
\[
u(x,t)=U\,e^{i(\omega t-kx)},\qquad U\in\mathbb{C}.
\]
Alors
\[
\partial_{tt}u=-\omega^2\,Ue^{i(\omega t-kx)},\qquad
\partial_{xx}u=-k^2\,Ue^{i(\omega t-kx)}.
\]
En substituant dans l’équation, on obtient la condition de non-trivialité $U\neq 0$ :
\[
-\omega^2=-c^2k^2
\qquad\Longrightarrow\qquad
\boxed{\;\omega^2=c^2k^2\;}
\quad\text{donc}\quad
\omega=\pm ck.
\]
On retient souvent $\omega\ge 0$ et on laisse $k$ porter le signe de la direction de propagation.

\paragraph{Notations usuelles.}
\[
k=\frac{2\pi}{\lambda},\qquad \omega=\frac{2\pi}{T},\qquad \nu=\frac{1}{T}=\frac{\omega}{2\pi}.
\]
La phase est $\Phi(x,t)=\omega t-kx+\varphi$ et l’onde réelle correspondante est
\[
u(x,t)=A\cos(\omega t-kx+\varphi),
\qquad A=|U|.
\]

% =========================================================
\subsection*{Vitesse de phase : déduction à partir des surfaces de phase constante}
\addcontentsline{toc}{subsection}{Vitesse de phase}
% =========================================================

Les points de phase constante vérifient $\Phi(x,t)=\text{cste}$ :
\[
\omega t-kx=\text{cste}.
\]
En différentiant,
\[
\omega\,\dd t-k\,\dd x=0
\qquad\Longrightarrow\qquad
\frac{\dd x}{\dd t}=\frac{\omega}{k}.
\]
On définit ainsi la \textbf{vitesse de phase}
\[
\boxed{\;v_{\mathrm{ph}}=\frac{\omega}{k}\;}
\]
et, dans le cas non dispersif $\omega=ck$, on trouve $v_{\mathrm{ph}}=c$.

\begin{pitfallbox}
La vitesse de phase n’est pas toujours une vitesse de transport d’énergie / d’information.
Dans certains milieux dispersifs, on peut même obtenir $v_{\mathrm{ph}}>c$ sans contradiction avec la causalité,
car le transport du signal est gouverné par la vitesse de groupe.
\end{pitfallbox}

% =========================================================
\subsection*{Relation de dispersion (cadre général) et dispersion}
\addcontentsline{toc}{subsection}{Relation de dispersion et dispersion}
% =========================================================

Dans un milieu général (ou pour une EDP plus complexe), une onde plane harmonique
$e^{i(\omega t-kx)}$ est solution si et seulement si $(\omega,k)$ vérifient une relation
\[
\boxed{\;\omega=\omega(k)\;}
\]
appelée \textbf{relation de dispersion}.
Le milieu est dit
\begin{itemize}
  \item \textbf{non dispersif} si $\omega(k)$ est affine (typiquement $\omega=ck$) ;
  \item \textbf{dispersif} sinon (différentes composantes en $k$ se propagent différemment).
\end{itemize}

% =========================================================
\subsection*{Paquets d’ondes : construction et vitesse de groupe}
\addcontentsline{toc}{subsection}{Paquets d’ondes et vitesse de groupe}
% =========================================================

Un signal réaliste est une superposition de modes :
\[
u(x,t)=\int_{\mathbb{R}} A(k)\,e^{i(\omega(k)t-kx)}\,\dd k,
\]
où $A(k)$ est concentré autour d’un $k_0$ (signal quasi-monochromatique).
On écrit $k=k_0+\kappa$ et on développe $\omega(k)$ au voisinage de $k_0$ :
\[
\omega(k)=\omega_0+\omega'_0\,\kappa+\frac12\omega''_0\,\kappa^2+\cdots,
\qquad
\omega_0=\omega(k_0),\ \omega'_0=\omega'(k_0).
\]
Ainsi,
\[
u(x,t)=e^{i(\omega_0 t-k_0 x)}\int_{\mathbb{R}} A(k_0+\kappa)\,
\exp\!\Big(i\big[\omega'_0 t-x\big]\kappa+\frac{i}{2}\omega''_0 t\,\kappa^2+\cdots\Big)\,\dd\kappa.
\]

\paragraph{Premier ordre : propagation de l’enveloppe.}
Si le spectre est suffisamment étroit (ou si l’on se place à des temps pas trop grands),
on néglige les termes $\mathcal{O}(\kappa^2)$ et on obtient
\[
u(x,t)\simeq e^{i(\omega_0 t-k_0 x)}\int_{\mathbb{R}} A(k_0+\kappa)\,
e^{i(\omega'_0 t-x)\kappa}\,\dd\kappa.
\]
En posant
\[
\mathcal{G}(X)=\int_{\mathbb{R}} A(k_0+\kappa)\,e^{-iX\kappa}\,\dd\kappa
\quad\text{(transformée de Fourier)},
\]
on reconnaît
\[
u(x,t)\simeq e^{i(\omega_0 t-k_0 x)}\,\mathcal{G}\big(x-\omega'_0 t\big).
\]
L’enveloppe $\mathcal{G}$ se déplace donc à la vitesse
\[
\boxed{\;v_{\mathrm{g}}=\omega'(k_0)=\frac{\dd\omega}{\dd k}(k_0)\;}
\]
appelée \textbf{vitesse de groupe}.

\begin{intuitionbox}
Dans l’approximation quasi-monochromatique, on obtient une onde
\[
u(x,t)\approx \underbrace{\mathcal{G}(x-v_{\mathrm{g}}t)}_{\text{enveloppe}}
\;\times\;
\underbrace{\cos(\omega_0 t-k_0 x)}_{\text{porteuse}},
\]
où la porteuse est associée à $v_{\mathrm{ph}}$ et l’enveloppe à $v_{\mathrm{g}}$.
\end{intuitionbox}

\paragraph{Second ordre : étalement du paquet (dispersion).}
Le terme $\frac12\omega''_0\kappa^2$ induit une déformation de l’enveloppe au cours du temps :
le paquet s’étale si $\omega''(k_0)\neq 0$.
Pour illustrer cela, prenons un spectre gaussien
\[
A(k)=A_0\,\exp\!\Big(-\frac{(k-k_0)^2}{2\sigma_k^2}\Big).
\]
Alors l’intégrale précédente (au second ordre) reste calculable et montre que
la largeur spatiale croît typiquement comme
\[
\sigma_x(t)\sim \sigma_x(0)\sqrt{1+\big(\omega''_0\,t\,\sigma_k^2\big)^2},
\qquad \sigma_x(0)\sim \frac{1}{\sigma_k}.
\]
Ainsi, \textbf{la dispersion} ($\omega''_0\neq 0$) implique \textbf{l’étalement} d’un paquet d’ondes.

% =========================================================
\subsection*{Résumé mathématique : phase, groupe, dispersion}
\addcontentsline{toc}{subsection}{Résumé : phase, groupe, dispersion}
% =========================================================

\begin{itemize}
  \item \textbf{Vitesse de phase} :
  \[
  v_{\mathrm{ph}}=\frac{\omega}{k}
  \quad\text{(vitesse des surfaces de phase constante)}.
  \]
  \item \textbf{Vitesse de groupe} :
  \[
  v_{\mathrm{g}}=\frac{\dd\omega}{\dd k}(k_0)
  \quad\text{(vitesse de l’enveloppe d’un paquet centré en $k_0$)}.
  \]
  \item \textbf{Non-dispersion} : $\omega(k)=ck$ $\Rightarrow v_{\mathrm{ph}}=v_{\mathrm{g}}=c$ et pas d’étalement.
  \item \textbf{Dispersion} : $\omega''(k_0)\neq 0$ $\Rightarrow$ étalement du paquet et, en général, $v_{\mathrm{ph}}\neq v_{\mathrm{g}}$.
\end{itemize}
