\subsection*{Probabilités}

\subsubsection{Désintegration d'une particule radioative}

1. Soit $F(t)$ la fonction de répartition: $F(t) = \mathbb{P}(T \leq t)$. On a:

$$
F(t) = \int_0^t p(t') dt'
$$
ou alors, en dérivant: $p(t) = F'(t)$. (1)

2. La Désintegration radiocactive est un proccessus Markovien, donc 
$$\mathbb{P}(T \leq t+dt|T > t) = \gamma dt$$ i.e.
$$\mathbb{P}(t <T \leq t+dt|T > t) = \gamma dt \mathbb{P}(T > t) = \gamma dt (1-F(t))$$.

3. On peut écrire $\mathbb{P}(t<T\leq t+dt) = F(t+dt) - F(t)$. Et donc, 
$$
\frac{F(t+dt)-F(t)}{dt} = \gamma (1-F(t))
$$

en faisant $dt \to 0$, on obtient l'équation différentielle:
$$F'(t) = \gamma (1-F(t)) \quad \text{(2)}
$$

En combinant (1) et (2), on obtient l'équation différentielle:
$$p(t) = -\gamma p(t)$$
$p(0) = \gamma$, puisque $p(0)dt = \mathbb{P}(0\leq t \leq dt) = \gamma dt$. Et donc, la solution de cette équation différentielle est:
$$p(t) = \gamma e^{-\gamma t}$$

4. Vérifions que $p(t)$ est bien une densité de probabilité:
$$
\int_0^{+\infty} p(t) dt = \int_0^{+\infty} \gamma e^{-\gamma t} dt = \left[-e^{-\gamma t}\right]_0^{+\infty} = 1
$$

5. On cherche $\langle t \rangle$ et $\Delta t = \sqrt{\langle t^2 \rangle - \langle t \rangle^2}$.

$$
\langle t \rangle = \int_0^{+\infty} t p(t) dt = \int_0^{+\infty} t \gamma e^{-\gamma t} dt = \left[-t e^{-\gamma t}\right]_0^{+\infty} = \frac{1}{\gamma}
$$

$$
\langle t^2 \rangle = \int_0^{+\infty} t^2 p(t) dt = \int_0^{+\infty} t^2 \gamma e^{-\gamma t} dt = \frac{2}{\gamma^2}
$$

et donc, $$\Delta t = \sqrt{\langle t^2 \rangle - \langle t \rangle^2} = \frac{1}{\gamma}$$.

\pagebreak

\subsubsection*{Distribution Gaussienne}

\pagebreak

\subsection*{Inteférences quantiques avec fentes de Young}

1. Pc. de superposition: si $Psi_1$ et $Psi_2$ sont deux fonctions d'onde qui écrivent l'état du système, alors toute combinaison linéaire $\Psi = c_1 \Psi_1 + c_2 \Psi_2$ avec $c_1, c_2 \in \mathbb{C}^2$ décrit aussi un état du système. \\

2. Les ondes issues de $F_1$ et $F_2$ prennent la forme: $\Psi_i = \Psi_0 e^{-i(kr_i - \omega t)}$

On a alors 
$$\Psi_{tot} = \Psi_1 + \Psi_2 = \Psi_0 e^{-i(kr_1 - \omega t)} + \Psi_0 e^{-i(kr_2 - \omega t)} = \Psi_0 e^{-i\omega t}(e^{-ikr_1} + e^{-ikr_2})$$

$$= \Psi_0 e^{-i\omega t} e^{-ik\frac{r_1+r_2}{2}}(e^{ik\frac{r_1-r_2}{2}} + e^{-ik\frac{r_1-r_2}{2}}) = 2\Psi_0 e^{-i\omega t} e^{-ik\frac{r_1+r_2}{2}}\cos\left(k \frac{r_1-r_2}{2}\right)$$

La densité de probabilité s'écrit donc:
$$|\Psi_{tot}|^2 = 4|\Psi_0|^2 \cos^2\left(k \frac{r_1-r_2}{2}\right)$$

2. On cherche $\Delta\Phi = k(r_1-r_2)$ avec $r_i = \sqrt{(x-x_i)^2 + D^2}$.

On a $|x-x_i| \ll D$, donc on peut faire un développement limité:

$$
r_i = D\sqrt{1 + \frac{(x-x_i)^2}{D^2}}
$$

i.e. 

$$
r_i \approx D \left(1 + \frac{(x-x_i)^2}{2D^2}\right) = D + \frac{(x-x_i)^2}{2D}
$$

Donc,

$$
\Delta \Phi = k(r_1 - r_2) = \frac{h}{2D}2ax = \frac{hax}{D}
$$.

et donc, $|\Psi_{tot}|^2 = 4|\Psi_0|^2 \cos^2\left(\frac{hax}{D}\right)$.

On cherche l'interfrange $\delta$ tel que $\frac{ha(x+\delta)}{D} = \frac{hax}{D} + 2\pi$, i.e. $\delta = \frac{2\pi D}{ha} = \frac{\lambda D}{a}$. \\

4. Sur l’écran, on observe une figure d’interférences caractérisée par une alternance régulière de franges brillantes et sombres. 
La densité de probabilité de présence n’est donc pas uniforme, mais présente des maxima et des minima en fonction de la position $x$.

Ce résultat s’explique par le principe de superposition quantique : la fonction d’onde totale sur l’écran est la somme des amplitudes associées aux deux fentes. 
Selon la position d’observation, les deux contributions peuvent interférer constructivement (franges brillantes) ou destructivement (franges sombres), en fonction de la différence de phase $\Delta\phi$.

Ce comportement diffère radicalement du cas de particules classiques (balles), pour lesquelles on observerait simplement la somme des intensités issues de chaque fente, sans franges d’interférence.
Il diffère également du cas purement ondulatoire classique dans son interprétation : ici, ce n’est pas l’intensité du champ qui interfère directement, mais l’amplitude de probabilité associée à une particule individuelle.

Ainsi, bien que chaque particule soit détectée ponctuellement sur l’écran, la répétition de l’expérience révèle une figure d’interférences caractéristique d’un comportement ondulatoire.
Ce phénomène illustre la dualité onde-particule et constitue une manifestation typique de la nature quantique de la matière.